% Options for packages loaded elsewhere
\PassOptionsToPackage{unicode}{hyperref}
\PassOptionsToPackage{hyphens}{url}
\PassOptionsToPackage{dvipsnames,svgnames,x11names}{xcolor}
%
\documentclass[
  letterpaper,
  DIV=11,
  numbers=noendperiod]{scrartcl}

\usepackage{amsmath,amssymb}
\usepackage{iftex}
\ifPDFTeX
  \usepackage[T1]{fontenc}
  \usepackage[utf8]{inputenc}
  \usepackage{textcomp} % provide euro and other symbols
\else % if luatex or xetex
  \usepackage{unicode-math}
  \defaultfontfeatures{Scale=MatchLowercase}
  \defaultfontfeatures[\rmfamily]{Ligatures=TeX,Scale=1}
\fi
\usepackage{lmodern}
\ifPDFTeX\else  
    % xetex/luatex font selection
\fi
% Use upquote if available, for straight quotes in verbatim environments
\IfFileExists{upquote.sty}{\usepackage{upquote}}{}
\IfFileExists{microtype.sty}{% use microtype if available
  \usepackage[]{microtype}
  \UseMicrotypeSet[protrusion]{basicmath} % disable protrusion for tt fonts
}{}
\makeatletter
\@ifundefined{KOMAClassName}{% if non-KOMA class
  \IfFileExists{parskip.sty}{%
    \usepackage{parskip}
  }{% else
    \setlength{\parindent}{0pt}
    \setlength{\parskip}{6pt plus 2pt minus 1pt}}
}{% if KOMA class
  \KOMAoptions{parskip=half}}
\makeatother
\usepackage{xcolor}
\setlength{\emergencystretch}{3em} % prevent overfull lines
\setcounter{secnumdepth}{-\maxdimen} % remove section numbering
% Make \paragraph and \subparagraph free-standing
\ifx\paragraph\undefined\else
  \let\oldparagraph\paragraph
  \renewcommand{\paragraph}[1]{\oldparagraph{#1}\mbox{}}
\fi
\ifx\subparagraph\undefined\else
  \let\oldsubparagraph\subparagraph
  \renewcommand{\subparagraph}[1]{\oldsubparagraph{#1}\mbox{}}
\fi


\providecommand{\tightlist}{%
  \setlength{\itemsep}{0pt}\setlength{\parskip}{0pt}}\usepackage{longtable,booktabs,array}
\usepackage{calc} % for calculating minipage widths
% Correct order of tables after \paragraph or \subparagraph
\usepackage{etoolbox}
\makeatletter
\patchcmd\longtable{\par}{\if@noskipsec\mbox{}\fi\par}{}{}
\makeatother
% Allow footnotes in longtable head/foot
\IfFileExists{footnotehyper.sty}{\usepackage{footnotehyper}}{\usepackage{footnote}}
\makesavenoteenv{longtable}
\usepackage{graphicx}
\makeatletter
\def\maxwidth{\ifdim\Gin@nat@width>\linewidth\linewidth\else\Gin@nat@width\fi}
\def\maxheight{\ifdim\Gin@nat@height>\textheight\textheight\else\Gin@nat@height\fi}
\makeatother
% Scale images if necessary, so that they will not overflow the page
% margins by default, and it is still possible to overwrite the defaults
% using explicit options in \includegraphics[width, height, ...]{}
\setkeys{Gin}{width=\maxwidth,height=\maxheight,keepaspectratio}
% Set default figure placement to htbp
\makeatletter
\def\fps@figure{htbp}
\makeatother

\KOMAoption{captions}{tableheading}
\makeatletter
\makeatother
\makeatletter
\makeatother
\makeatletter
\@ifpackageloaded{caption}{}{\usepackage{caption}}
\AtBeginDocument{%
\ifdefined\contentsname
  \renewcommand*\contentsname{Table of contents}
\else
  \newcommand\contentsname{Table of contents}
\fi
\ifdefined\listfigurename
  \renewcommand*\listfigurename{List of Figures}
\else
  \newcommand\listfigurename{List of Figures}
\fi
\ifdefined\listtablename
  \renewcommand*\listtablename{List of Tables}
\else
  \newcommand\listtablename{List of Tables}
\fi
\ifdefined\figurename
  \renewcommand*\figurename{Figure}
\else
  \newcommand\figurename{Figure}
\fi
\ifdefined\tablename
  \renewcommand*\tablename{Table}
\else
  \newcommand\tablename{Table}
\fi
}
\@ifpackageloaded{float}{}{\usepackage{float}}
\floatstyle{ruled}
\@ifundefined{c@chapter}{\newfloat{codelisting}{h}{lop}}{\newfloat{codelisting}{h}{lop}[chapter]}
\floatname{codelisting}{Listing}
\newcommand*\listoflistings{\listof{codelisting}{List of Listings}}
\makeatother
\makeatletter
\@ifpackageloaded{caption}{}{\usepackage{caption}}
\@ifpackageloaded{subcaption}{}{\usepackage{subcaption}}
\makeatother
\makeatletter
\@ifpackageloaded{tcolorbox}{}{\usepackage[skins,breakable]{tcolorbox}}
\makeatother
\makeatletter
\@ifundefined{shadecolor}{\definecolor{shadecolor}{rgb}{.97, .97, .97}}
\makeatother
\makeatletter
\makeatother
\makeatletter
\makeatother
\ifLuaTeX
  \usepackage{selnolig}  % disable illegal ligatures
\fi
\IfFileExists{bookmark.sty}{\usepackage{bookmark}}{\usepackage{hyperref}}
\IfFileExists{xurl.sty}{\usepackage{xurl}}{} % add URL line breaks if available
\urlstyle{same} % disable monospaced font for URLs
\hypersetup{
  pdftitle={Lista 1},
  pdfauthor={Tailine J. S. Nonato},
  colorlinks=true,
  linkcolor={blue},
  filecolor={Maroon},
  citecolor={Blue},
  urlcolor={Blue},
  pdfcreator={LaTeX via pandoc}}

\title{Lista 1}
\usepackage{etoolbox}
\makeatletter
\providecommand{\subtitle}[1]{% add subtitle to \maketitle
  \apptocmd{\@title}{\par {\large #1 \par}}{}{}
}
\makeatother
\subtitle{Estatística Matemática}
\author{Tailine J. S. Nonato}
\date{April 14, 2025}

\begin{document}
\maketitle
\ifdefined\Shaded\renewenvironment{Shaded}{\begin{tcolorbox}[interior hidden, sharp corners, breakable, borderline west={3pt}{0pt}{shadecolor}, boxrule=0pt, enhanced, frame hidden]}{\end{tcolorbox}}\fi

\hypertarget{exercuxedcio-1}{%
\subsection{Exercício 1}\label{exercuxedcio-1}}

Se \(\Omega = \{C, R\} \times \{C, R\}\), determine o conjunto potência
\(P(\Omega)\).

\hypertarget{exercuxedcio-2}{%
\subsection{Exercício 2}\label{exercuxedcio-2}}

Se \(\mathcal{F}\) é uma \(\sigma\)-álgebra, verifique que:

\begin{enumerate}
\def\labelenumi{(\alph{enumi})}
\item
  Se \(\{A_i\}_{i=1}^n \subset \mathcal{F}\) então
  \(\bigcup_{i=1}^n A_i \in \mathcal{F}\) e
  \(\bigcap_{i=1}^n A_i \in \mathcal{F}\).
\item
  Se \(\{A_i\}_{i=1}^\infty \subset \mathcal{F}\) então
  \(\bigcap_{i=1}^\infty A_i \in \mathcal{F}\).
\item
  \(A, B \in \mathcal{F}\) então \(A \cap B^c \in \mathcal{F}\).
\end{enumerate}

\hypertarget{exercuxedcio-3}{%
\subsection{Exercício 3}\label{exercuxedcio-3}}

Seja \(I\) um conjunto de índices. Verifique que: se
\(\{\mathcal{F}_i\}_{i\in I}\) é uma família de \(\sigma\)-álgebras,
então \(\bigcap_{i\in I}\mathcal{F}_i\) é uma \(\sigma\)-álgebra.

\hypertarget{exercuxedcio-4}{%
\subsection{Exercício 4}\label{exercuxedcio-4}}

Considere \(\Omega = \{a, b, c\}\) e as coleções
\(\mathcal{F}_1 = \{0, \Omega, \{a\}, \{b, c\}\}\) e
\(\mathcal{F}_2 = \{0, \Omega, \{c\}, \{a, b\}\}\). As coleções
\(\mathcal{F}_1\) e \(\mathcal{F}_2\) são \(\sigma\)-álgebras?
\(\mathcal{F}_1 \cup \mathcal{F}_2\) é uma \(\sigma\)-álgebra?

\textbf{Resposta:} (a) Sim, (b) Não.

\hypertarget{exercuxedcio-5}{%
\subsection{Exercício 5}\label{exercuxedcio-5}}

Sejam \(\Omega = \{1, 2, 3, 4, 5, 6\}\), \(A =\) ``o resultado é um
número par'' e \(B = A - \{6\}\). Encontre as \(\sigma\)-álgebras
\(\sigma(\{A\})\) e \(\sigma(\{A, B\})\), e as compare.

\hypertarget{exercuxedcio-6}{%
\subsection{Exercício 6}\label{exercuxedcio-6}}

Obtenha a \(\sigma\)-álgebra gerada pela classe
\(\mathcal{C} = \{0, 1\}, \{1, 2\}\) se:

\begin{enumerate}
\def\labelenumi{(\alph{enumi})}
\item
  \(\Omega = \{0, 1, 2\}\).
\item
  \(\Omega = \{0, 1, 2, 3\}\).
\end{enumerate}

\hypertarget{exercuxedcio-7}{%
\subsection{Exercício 7}\label{exercuxedcio-7}}

Obtenha a \(\sigma\)-álgebra de \(\Omega = \{1, 2, 3, 4, 5\}\) gerada
por:

\begin{enumerate}
\def\labelenumi{(\alph{enumi})}
\item
  \(\mathcal{C}_1 = \{\{2\}\}\).
\item
  \(\mathcal{C}_2 = \{\{1, 2\}\}\).
\item
  \(\mathcal{C}_3 = \{\{1, 2, 3\}\}\).
\item
  \(\mathcal{C}_4 = \{\{1, 2\}, \{1, 3\}\}\).
\item
  \(\mathcal{C}_5 = \{\{1\}, \{2, 3\}\}\).
\end{enumerate}

\hypertarget{exercuxedcio-8}{%
\subsection{Exercício 8}\label{exercuxedcio-8}}

Construa a menor \(\sigma\)-álgebra em \([0,1]\), contendo o subconjunto
\([1/4,3/4]\).

\hypertarget{exercuxedcio-9}{%
\subsection{Exercício 9}\label{exercuxedcio-9}}

Sejam \(A_{1},A_{2},\ldots\) eventos aleatórios. Verifique que:

\begin{enumerate}
\def\labelenumi{(\alph{enumi})}
\item
  \(\mathbb{P}\Big{(}\bigcap_{k=1}^{n}A_{k}\Big{)}\geqslant 1-\sum_{k=1}^{n}\mathbb{P}(A_{k}^{\varepsilon})\).
\item
  Se \(\mathbb{P}(A_{k})\geqslant 1-\varepsilon\) para
  \(k=1,\ldots,n\in\varepsilon>0\), então
  \(\mathbb{P}\Big{(}\bigcap_{k=1}^{n}A_{k}\Big{)}\geqslant 1-ne\).
\item
  \(\mathbb{P}\Big{(}\bigcap_{k=1}^{\infty}A_{k}\Big{)}\geqslant 1-\sum_{k=1}^{\infty}\mathbb{P}(A_{k}^{\varepsilon})\).
\end{enumerate}

\textbf{Resposta:} (a) Use a sub-aditividade de \(\mathbb{P}\), (b) Use
o Item (a), (c) Use a sub-aditividade de \(\mathbb{P}\).

\hypertarget{exercuxedcio-10}{%
\subsection{Exercício 10}\label{exercuxedcio-10}}

Verifique as seguintes propriedades:

\begin{enumerate}
\def\labelenumi{(\alph{enumi})}
\item
  Se \(\mathbb{P}(A_{n})=0\) para \(n=1,2,\ldots\), então
  \(\mathbb{P}\Big{(}\bigcup_{n=1}^{\infty}A_{n}\Big{)}=0\).
\item
  Se \(\mathbb{P}(A_{n})=1\) para \(n=1,2,\ldots\), então
  \(\mathbb{P}\Big{(}\bigcap_{n=1}^{\infty}A_{n}\Big{)}=1\).
\end{enumerate}

\textbf{Resposta:} Para (a) e (b) use a sub-aditividade de
\(\mathbb{P}\).

\hypertarget{exercuxedcio-11}{%
\subsection{Exercício 11}\label{exercuxedcio-11}}

Um casal é escolhido ao acaso e o número de filhos e filhas perguntado.
Considerando o evento
\(A=\text{``um casal não tem filhos''}\in\mathbb{P}(\{\omega\})=1/2^{x+y+2}\)
para todo \(\omega=(x,y)\in\Omega\), determine \(\mathbb{P}(A)\).

\textbf{Resposta:} \(1/2\).

\hypertarget{exercuxedcio-12}{%
\subsection{Exercício 12}\label{exercuxedcio-12}}

Uma moeda é lançada \(n\) vezes, \(n\geqslant 2\). Qual é a
probabilidade de que, nestes \(n\) lançamentos, não apareçam \(2\) caras
seguidas?

\textbf{Resposta:} \(\frac{{n\choose 2}+2}{2^{n}}\).

\hypertarget{notauxe7uxe3o-de-conjuntos}{%
\subsection{Notação de Conjuntos}\label{notauxe7uxe3o-de-conjuntos}}

Seja \(\{A_{i}\}_{i=1}^{\infty}\subset\mathscr{F}\). Denotamos:

\begin{enumerate}
\def\labelenumi{(\Roman{enumi})}
\item
  \(A_{n}\uparrow A\iff A_{n}\subset A_{n+1}\ \forall n\geqslant 1\ \in\bigcup_{i=1}^{\infty}A_{i}=A\).
\item
  \(A_{n}\downarrow A\iff A_{n+1}\subset A_{n}\ \forall n\geqslant 1\ \in\bigcap_{i=1}^{\infty}A_{i}=A\).
\item
  \(\limsup_{n\to\infty}A_{n}\stackrel{{\mathrm{def.}}}{{:=}}\bigcap_{n=1}^{\infty}\bigcup_{m=n}^{\infty}A_{m}\).
\item
  \(\liminf_{n\to\infty}A_{n}\stackrel{{\mathrm{def.}}}{{:=}}\bigcup_{n=1}^{\infty}\bigcap_{m=n}^{\infty}A_{m}\).
\item
  \(A_{n}\to A\iff\limsup_{n\to\infty}A_{n}=\liminf_{n\to\infty}A_{n}=A\).
\end{enumerate}

\hypertarget{exercuxedcio-13}{%
\subsection{Exercício 13}\label{exercuxedcio-13}}

Seja \(\{A_{i}\}_{i=1}^{\infty}\subset\mathscr{F}\). Verifique que:

\begin{enumerate}
\def\labelenumi{(\alph{enumi})}
\item
  Se \(A_{n}\uparrow A\) então
  \(\lim_{n\to\infty}\mathbb{P}(A_{n})=\mathbb{P}(A)\in\mathbb{P}(A_{n})\leqslant \mathbb{P}(A_{n+1})\ \forall n\geqslant 1\).
\item
  Se \(A_{n}\downarrow A\) então
  \(\lim_{n\to\infty}\mathbb{P}(A_{n})=\mathbb{P}(A)\in\mathbb{P}(A_{n})\geqslant \mathbb{P}(A_{n+1})\ \forall n\geqslant 1\).
\item
  Se \(A_{n}\to A\) então
  \(A\in\mathscr{F}\in\lim_{n\to\infty}\mathbb{P}(A_{n})=\mathbb{P}(A)\).
\item
  Se \(\sum_{i=1}^{\infty}\mathbb{P}(A_{i})<\infty\) então
  \(\mathbb{P}\big{(}\limsup_{n\to\infty}A_{n}\big{)}=0\).
\end{enumerate}

\textbf{Resposta:} (a) Escreva \(A_n\) e \(A\) como união disjunta de
eventos e aplique a \(\sigma\)-aditividade de \(\mathbb{P}\),

\begin{enumerate}
\def\labelenumi{(\alph{enumi})}
\setcounter{enumi}{1}
\item
  Use o Item (a),
\item
  Use os Itens (a) e (b),
\item
  Use a sub-aditividade de \(\mathbb{P}\) e o fato de que {[}
  \lim\emph{\{n \to \infty\} \sum}\{i=n\}\^{}\{\infty\} \mathbb{P}(A\_i)
  = 0. {]}
\end{enumerate}

\hypertarget{exercuxedcio-14}{%
\subsection{Exercício 14}\label{exercuxedcio-14}}

Se \(A, B, C \in \mathcal{F}\), verifique que:

{[} \mathbb{P}(A \cup B \cup C) = \mathbb{P}(A) + \mathbb{P}(B) +
\mathbb{P}(C) - \mathbb{P}(A \cap B) - \mathbb{P}(A \cap C) -
\mathbb{P}(B \cap C) + \mathbb{P}(A \cap B \cap C). {]}

\hypertarget{exercuxedcio-15}{%
\subsection{Exercício 15}\label{exercuxedcio-15}}

Verifique que a aplicação \(A \mapsto \mathbb{P}(A|B)\) é uma medida de
probabilidade.

\hypertarget{exercuxedcio-16}{%
\subsection{Exercício 16}\label{exercuxedcio-16}}

Certo experimento consiste em lançar um dado equilibrado 2 vezes,
independentemente. Dado que os dois números sejam diferentes, qual é a
probabilidade de:

\begin{enumerate}
\def\labelenumi{(\alph{enumi})}
\item
  Pelo menos um dos números ser 6?
\item
  A soma dos números ser 8?
\end{enumerate}

\textbf{Resposta:} (a) 1/3, (b) 2/15.

\hypertarget{exercuxedcio-17}{%
\subsection{Exercício 17}\label{exercuxedcio-17}}

Verifique que:

\begin{enumerate}
\def\labelenumi{(\alph{enumi})}
\item
  Um evento \(A\) com \(\mathbb{P}(A) = 0\) é independente a qualquer
  outro evento \(B\).
\item
  Um evento \(A\) com \(\mathbb{P}(A) = 1\) é independente a qualquer
  outro evento \(B\).
\end{enumerate}

\hypertarget{exercuxedcio-18}{%
\subsection{Exercício 18}\label{exercuxedcio-18}}

Considere o circuito em série da figura abaixo, onde \(R_1\) e \(R_2\)
são componentes eletrônicos idênticos que permitem a passagem de
corrente elétrica, cuja probabilidade da corrente perpassar cada um é
\(p\). Determine a probabilidade da corrente sair de \(A\) e chegar a
\(B\).

\textbf{Resposta:} \(p^2\).

\hypertarget{exercuxedcio-19}{%
\subsection{Exercício 19}\label{exercuxedcio-19}}

Sejam \(A_1, \ldots, A_n\) eventos aleatórios independentes, com
\(\mathbb{P}(A_k) = p_k, k = 1, \ldots, n\). Obtenha a probabilidade de
ocorrência dos seguintes eventos, em termos das probabilidades \(p_k\):

\begin{enumerate}
\def\labelenumi{(\alph{enumi})}
\item
  A ocorrência de nenhum dos \(A_k\).
\item
  A ocorrência de pelo menos um dos \(A_k\).
\item
  A ocorrência de exatamente um dos \(A_k\).
\item
  A ocorrência de exatamente dois dos \(A_k\).
\item
  A ocorrência de todos os \(A_k\).
\item
  A ocorrência de, no máximo, \(n - 1\) dos \(A_k\).
\end{enumerate}

\textbf{Resposta:} (a) \(\prod_{k=1}^{n} (1 - p_k)\),\\
(b) \(1 - \prod_{k=1}^{n} (1 - p_k)\),\\
(c) \(\sum_{i=1}^{n} p_i \prod_{k \neq i} (1 - p_k)\),\\
(d) \(\sum_{i<j} p_i p_j \prod_{k \neq i,j} (1 - p_k)\),\\
(e) \(\prod_{k=1}^{n} p_k\),\\
(f) \(1 - \prod_{k=1}^{n} p_k\).

\hypertarget{exercuxedcio-20}{%
\subsection{Exercício 20}\label{exercuxedcio-20}}

Sejam \(A, B, C\) eventos definidos na figura

\begin{enumerate}
\def\labelenumi{(\alph{enumi})}
\tightlist
\item
  Os eventos \(A\) e \(C\) são independentes?\\
\item
  Os eventos \(A\) e \(B\) são independentes?
\end{enumerate}

\textbf{Resposta:} (a) Sim, (b) Não.

\hypertarget{exercuxedcio-21}{%
\subsection{Exercício 21}\label{exercuxedcio-21}}

Sejam \(A_1, A_2, \ldots, A_n\) e \(B_1, B_2, \ldots, B_n\) eventos
definidos em \((\Omega, \mathcal{F}, \mathbb{P})\). Para
\(j = 1, 2, \ldots, n\) suponha que \(B_j\) seja independente de
\(\bigcap_{i=1}^n A_i\) e que os \(B_j\)'s sejam disjuntos 2 a 2. Os
eventos \(\bigcup_{j=1}^n B_j\) e \(\bigcap_{i=1}^n A_i\) são
independentes?

\textbf{Resposta:} Sim.

\hypertarget{exercuxedcio-22}{%
\subsection{Exercício 22}\label{exercuxedcio-22}}

Seja \(\Omega = \{abc, acb, cab, cba, bca, bac, aaa, bbb, ccc\}\) com
\(\mathbb{P}(\{\omega\}) = 1/9\) para todo \(\omega \in \Omega\), e o
evento \(A_k = k\)-ésima letra é \(a\), \(k = 1, 2, 3\). Verifique que a
família \(\{A_1, A_2, A_3\}\) é 2 a 2 independente porém não
independente (coletivamente).

\hypertarget{exercuxedcio-23}{%
\subsection{Exercício 23}\label{exercuxedcio-23}}

Os coeficientes \(a\) e \(b\) da equação \(ax^2 + bx + 1 = 0\) são,
respectivamente, os resultados sucessivos de 2 lançamentos de um dado
equilibrado. Determine a probabilidade das raízes dessa equação serem
números reais.

\textbf{Resposta:} 19/36.

\hypertarget{exercuxedcio-24}{%
\subsection{Exercício 24}\label{exercuxedcio-24}}

Assuma que para cada cliente que solicita o cancelamento de um plano, a
companhia responsável o faça com probabilidade \(q\).

\begin{enumerate}
\def\labelenumi{(\alph{enumi})}
\item
  Se 4 clientes solicitam o cancelamento, qual a probabilidade de que a
  companhia cancele o plano de exatamente 2 clientes?
\item
  Qual a probabilidade de que sejam necessários exatamente 10
  solicitações negadas para que o primeiro cancelamento seja efetuado?
\end{enumerate}

\textbf{Resposta:} (a) \(6q^2(1 - q)^2\), (b) \((1 - q)^{10}q\).

\hypertarget{exercuxedcio-25}{%
\subsection{Exercício 25}\label{exercuxedcio-25}}

Numa população 10\% das pessoas são infectadas por um determinado vírus.
Um teste para detecção do vírus é eficiente em 95\% dos casos nos quais
os indivíduos são infectados, mas resulta em 4\% de resultados positivos
para os não infectados. Qual a probabilidade de que o teste de uma
pessoa dessa população dê resultado positivo?

\textbf{Resposta:} 0.131.

\hypertarget{exercuxedcio-26}{%
\subsection{Exercício 26}\label{exercuxedcio-26}}

Suponha que temos duas urnas: a primeira tem 4 bolas azuis e 6 bolas
brancas, a outra tem 7 azuis e 3 brancas. Lançamos um dado honesto, se
sair um número par, selecionamos ao acaso uma bola da primeira urna, se
for um número ímpar da segunda. Qual a probabilidade de selecionar uma
bola azul?

\textbf{Resposta:} 0.55.

\hypertarget{exercuxedcio-27}{%
\subsection{Exercício 27}\label{exercuxedcio-27}}

Considere uma urna que contem 1 bola vermelha, 4 bolas brancas e 3
azuis. Supondo que se efetuam extrações sem reposição de 2 bolas,
determine a probabilidade de retirar 1 bola vermelha na segunda
extração.

\textbf{Resposta:} 0.125.

\hypertarget{exercuxedcio-28}{%
\subsection{Exercício 28}\label{exercuxedcio-28}}

Um lote contém 15 peças, de onde 5 são defeituosas. Se um funcionário
extrai uma amostra de 5 peças aleatoriamente:

\begin{enumerate}
\def\labelenumi{(\alph{enumi})}
\item
  Qual é a probabilidade de que a amostra não contenha peças defeituosas
  se a escolha foi realizada com reposição?
\item
  Qual é a probabilidade de que a amostra não contenha peças
  defeituosas, se a escolha foi realizada sem reposição?
\end{enumerate}

\textbf{Resposta:} (a) 0.1317, (b) 0.0839.

\hypertarget{exercuxedcio-29}{%
\subsection{Exercício 29}\label{exercuxedcio-29}}

Suponha que a ocorrência de chuva (ou não) dependa das condições do
tempo do dia anterior. Assumamos que, se (não) chova hoje, choverá
amanhã com probabilidade (respectivamente q) p.~Sabendo que choveu hoje,
calcule a probabilidade de chover depois de amanhã.

\textbf{Resposta:} \(p^2 + q(1 - p)\).



\end{document}
