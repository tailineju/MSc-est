% Options for packages loaded elsewhere
% Options for packages loaded elsewhere
\PassOptionsToPackage{unicode}{hyperref}
\PassOptionsToPackage{hyphens}{url}
\PassOptionsToPackage{dvipsnames,svgnames,x11names}{xcolor}
%
\documentclass[
  letterpaper,
  DIV=11,
  numbers=noendperiod]{scrartcl}
\usepackage{xcolor}
\usepackage{amsmath,amssymb}
\setcounter{secnumdepth}{-\maxdimen} % remove section numbering
\usepackage{iftex}
\ifPDFTeX
  \usepackage[T1]{fontenc}
  \usepackage[utf8]{inputenc}
  \usepackage{textcomp} % provide euro and other symbols
\else % if luatex or xetex
  \usepackage{unicode-math} % this also loads fontspec
  \defaultfontfeatures{Scale=MatchLowercase}
  \defaultfontfeatures[\rmfamily]{Ligatures=TeX,Scale=1}
\fi
\usepackage{lmodern}
\ifPDFTeX\else
  % xetex/luatex font selection
\fi
% Use upquote if available, for straight quotes in verbatim environments
\IfFileExists{upquote.sty}{\usepackage{upquote}}{}
\IfFileExists{microtype.sty}{% use microtype if available
  \usepackage[]{microtype}
  \UseMicrotypeSet[protrusion]{basicmath} % disable protrusion for tt fonts
}{}
\makeatletter
\@ifundefined{KOMAClassName}{% if non-KOMA class
  \IfFileExists{parskip.sty}{%
    \usepackage{parskip}
  }{% else
    \setlength{\parindent}{0pt}
    \setlength{\parskip}{6pt plus 2pt minus 1pt}}
}{% if KOMA class
  \KOMAoptions{parskip=half}}
\makeatother
% Make \paragraph and \subparagraph free-standing
\makeatletter
\ifx\paragraph\undefined\else
  \let\oldparagraph\paragraph
  \renewcommand{\paragraph}{
    \@ifstar
      \xxxParagraphStar
      \xxxParagraphNoStar
  }
  \newcommand{\xxxParagraphStar}[1]{\oldparagraph*{#1}\mbox{}}
  \newcommand{\xxxParagraphNoStar}[1]{\oldparagraph{#1}\mbox{}}
\fi
\ifx\subparagraph\undefined\else
  \let\oldsubparagraph\subparagraph
  \renewcommand{\subparagraph}{
    \@ifstar
      \xxxSubParagraphStar
      \xxxSubParagraphNoStar
  }
  \newcommand{\xxxSubParagraphStar}[1]{\oldsubparagraph*{#1}\mbox{}}
  \newcommand{\xxxSubParagraphNoStar}[1]{\oldsubparagraph{#1}\mbox{}}
\fi
\makeatother


\usepackage{longtable,booktabs,array}
\usepackage{calc} % for calculating minipage widths
% Correct order of tables after \paragraph or \subparagraph
\usepackage{etoolbox}
\makeatletter
\patchcmd\longtable{\par}{\if@noskipsec\mbox{}\fi\par}{}{}
\makeatother
% Allow footnotes in longtable head/foot
\IfFileExists{footnotehyper.sty}{\usepackage{footnotehyper}}{\usepackage{footnote}}
\makesavenoteenv{longtable}
\usepackage{graphicx}
\makeatletter
\newsavebox\pandoc@box
\newcommand*\pandocbounded[1]{% scales image to fit in text height/width
  \sbox\pandoc@box{#1}%
  \Gscale@div\@tempa{\textheight}{\dimexpr\ht\pandoc@box+\dp\pandoc@box\relax}%
  \Gscale@div\@tempb{\linewidth}{\wd\pandoc@box}%
  \ifdim\@tempb\p@<\@tempa\p@\let\@tempa\@tempb\fi% select the smaller of both
  \ifdim\@tempa\p@<\p@\scalebox{\@tempa}{\usebox\pandoc@box}%
  \else\usebox{\pandoc@box}%
  \fi%
}
% Set default figure placement to htbp
\def\fps@figure{htbp}
\makeatother





\setlength{\emergencystretch}{3em} % prevent overfull lines

\providecommand{\tightlist}{%
  \setlength{\itemsep}{0pt}\setlength{\parskip}{0pt}}



 


\KOMAoption{captions}{tableheading}
\makeatletter
\@ifpackageloaded{caption}{}{\usepackage{caption}}
\AtBeginDocument{%
\ifdefined\contentsname
  \renewcommand*\contentsname{Table of contents}
\else
  \newcommand\contentsname{Table of contents}
\fi
\ifdefined\listfigurename
  \renewcommand*\listfigurename{List of Figures}
\else
  \newcommand\listfigurename{List of Figures}
\fi
\ifdefined\listtablename
  \renewcommand*\listtablename{List of Tables}
\else
  \newcommand\listtablename{List of Tables}
\fi
\ifdefined\figurename
  \renewcommand*\figurename{Figure}
\else
  \newcommand\figurename{Figure}
\fi
\ifdefined\tablename
  \renewcommand*\tablename{Table}
\else
  \newcommand\tablename{Table}
\fi
}
\@ifpackageloaded{float}{}{\usepackage{float}}
\floatstyle{ruled}
\@ifundefined{c@chapter}{\newfloat{codelisting}{h}{lop}}{\newfloat{codelisting}{h}{lop}[chapter]}
\floatname{codelisting}{Listing}
\newcommand*\listoflistings{\listof{codelisting}{List of Listings}}
\makeatother
\makeatletter
\makeatother
\makeatletter
\@ifpackageloaded{caption}{}{\usepackage{caption}}
\@ifpackageloaded{subcaption}{}{\usepackage{subcaption}}
\makeatother
\usepackage{bookmark}
\IfFileExists{xurl.sty}{\usepackage{xurl}}{} % add URL line breaks if available
\urlstyle{same}
\hypersetup{
  pdftitle={Resumo},
  pdfauthor={Tailine J. S. Nonato},
  colorlinks=true,
  linkcolor={blue},
  filecolor={Maroon},
  citecolor={Blue},
  urlcolor={Blue},
  pdfcreator={LaTeX via pandoc}}


\title{Resumo}
\usepackage{etoolbox}
\makeatletter
\providecommand{\subtitle}[1]{% add subtitle to \maketitle
  \apptocmd{\@title}{\par {\large #1 \par}}{}{}
}
\makeatother
\subtitle{Análise de Sobrevivência}
\author{Tailine J. S. Nonato}
\date{June 18, 2025}
\begin{document}
\maketitle


\maketitle

\section*{Modelo de Log-Sobrevivência Proporcional}

O modelo é definido como: {[} \log S(t \mid \mathbf{x}) =
g(\mathbf{x}\^{}\top \beta) \cdot \log S\_0(t) {]} Ou, equivalentemente:
{[} S(t \mid \mathbf{x}) =
S\_0(t)\textsuperscript{\{g(\mathbf{x}}\top \beta)\} {]}

\subsection*{i) Equivalência com o modelo de riscos proporcionais}

O modelo de Cox (riscos proporcionais) é: {[} h(t \mid \mathbf{x}) =
h\_0(t) \cdot \exp(\mathbf{x}\^{}\top \beta) {]}

Queremos mostrar que o modelo de log-sobrevivência proporcional implica
riscos proporcionais sob certas condições.

\subsubsection*{Demonstração:}

Assuma que: {[} S(t \mid \mathbf{x}) =
\left[ S_0(t) \right]\textsuperscript{\{\exp(\mathbf{x}}\top \beta)\}
{]} A função de risco é dada por: {[} h(t \mid \mathbf{x}) =
\frac{f(t \mid \mathbf{x})}{S(t \mid \mathbf{x})} = -\frac{d}{dt}
\log S(t \mid \mathbf{x}) {]} Como: {[} \log S(t \mid \mathbf{x}) =
\exp(\mathbf{x}\^{}\top \beta) \cdot \log S\_0(t) {]} Temos: {[} h(t
\mid \mathbf{x}) = \exp(\mathbf{x}\^{}\top \beta) \cdot h\_0(t) {]}
\textbf{Conclusão:} Isso é exatamente o modelo de riscos proporcionais.

\subsection*{ii) Modelo Binomial Negativa (2,0) não possui riscos proporcionais}

Considere \(T \sim \text{Binomial Negativa}(r = 2, p)\). A função de
sobrevivência é: {[} S(t) = P(T \textgreater{} t) = (1 - p)\^{}\{t +
1\}(1 + (t + 1)p) {]}

Aplicando o modelo log-sobrevivência proporcional: {[} S(t \mid x) =
S\_0(t)\^{}\{g(x)\} =
\left[ (1 - p)^{t+1}(1 + (t+1)p) \right]\^{}\{g(x)\} {]}

A função de risco discreta é: {[} h(t \mid x) =
\frac{S(t \mid x) - S(t+1 \mid x)}{S(t \mid x)} {]}

Como o fator \(g(x)\) entra de forma não linear, especialmente em termos
como \((1 + (t+1)p)^{g(x)}\), o quociente de riscos depende de \(t\).

\textbf{Conclusão:} O modelo \textbf{não} possui a propriedade de riscos
proporcionais.

\subsection*{Bônus: Modelo Geométrico possui riscos proporcionais}

Se \(T \sim \text{Geom}(p)\), então: {[} S\_0(t) = (1 - p)\^{}\{t + 1\}
{]} Logo: {[} S(t \mid x) = S\_0(t)\^{}\{g(x)\} = (1 - p)\^{}\{g(x)(t +
1)\} {]}

A função de risco discreta é: {[} h(t \mid x) = 1 -
\frac{S(t + 1 \mid x)}{S(t \mid x)} = 1 -
\frac{(1 - p)^{g(x)(t+2)}}{(1 - p)^{g(x)(t+1)}} = 1 - (1 -
p)\^{}\{g(x)\} {]}

\textbf{Conclusão:} A função de risco depende apenas de \(x\) (via
\(g(x)\)), e não de \(t\).
\textbf{Portanto, o modelo geométrico apresenta riscos proporcionais.}




\end{document}
